\chapter{Activités}
\label{chap:activités}

Les activités sont toutes les tâches à effectuer pour réaliser le projet. Ces tâches viennent directement des objectifs et elles sont organisées en 3 niveaux afin de faciliter l'intégration à Gitlab.

   \section{Cahier des charges}
   \label{sec:cdc}

      Le cahier des charges doit être rédigé par l'étudiant et validé par tous les participants du projet. Il sera aussi défendu devant la classe afin de présenter le projet à tout le monde.

      \subsection*{Chapitre 1 : Contexte}
         Must be done:
         \begin{itemize}
            \item Ecrire le chapitre contexte [2]
         \end{itemize}
      \subsection*{Chapitre 2 : Objectifs}
         Must be done:
         \begin{itemize}
            \item Ecrire le chapitre objectifs [1]
         \end{itemize}
      \subsection*{Chapitre 3 : Activités}
         Must be done:
         \begin{itemize}
            \item Ecrire le chapitre activités [2]
         \end{itemize}
      \subsection*{Chapitre 4 : Planning}
         Must be done:
         \begin{itemize}
            \item Ecrire le chapitre planning [1]
            \item Créer le diagramme de Gantt [3]
         \end{itemize}
      \subsection*{Défense du cahier des charges}
         Must be done:
         \begin{itemize}
            \item Utiliser et adapter un template latex pour le cahier des charges [2]
            \item Publier le cahier des charges sur le gitlab [2]
            \item Créer une présentation [3]
         \end{itemize}
         
         \section{Reproduction du modèle SVM}
         \label{sec:svm}
         
         Le modèle SVM est un modèle de machine learning été utilisé dans l'article. Le reproduire permettra de poser les bases pour la suite et sera utilisé comme référence pour les prochains modèles.

         \subsection*{Prendre en main le modèle SVM}
            Must be done:
            \begin{itemize}
               \item Prendre les données en main [1]
               \item Expérimenter SVM [1]
            \end{itemize}
         \subsection*{Créer un modèle avec les \acrshort{smiles}}
            Must be done:
            \begin{itemize}
               \item Utiliser l'algorithme SVM [5]
               \item Comparer les résultats avec le projet de Mme Yerly [3]
               \item Documenter la démarche de reproduction [2]
            \end{itemize}

   \section{Base de données normalisées}
   \label{sec:base}

         Les données sont les fondations des modèles de machine learning, c'est pourquoi cet objectif sera le premier à être réalisé. Il est important de bien structurer les données afin de pouvoir les utiliser facilement pour les futurs développements.

      \subsection*{Analyser les données}
         Must be done:
         \begin{itemize}
            \item Identifier les données de l'article et de l'école [1]
            \item Repérer les données divergentes à l'aide du modèle [2]
            \item Documenter les sources de données [1]
         \end{itemize}

         Can be done:
         \begin{itemize}
            \item Identifier d'autres sources de données [1]
            \item Tester les autres sources de données [3]
         \end{itemize}
      \subsection*{Normaliser les données}
         Must be done:
         \begin{itemize}
            \item Définir les règles de normalisation [1]
            \item Définir la méthode de publication des données [1]
            \item Publier les données normalisées [2]
            \item Documenter l'infrasctucture et la normalisation [1]
         \end{itemize}
   
   \section{Modèle avec les SMILES}
   \label{sec:smiles}

            L'écriture \acrshort{smiles} encapsule les informations d'une liaison chimique. Afin de créer des modèles plus performants, on a besoin de ces informations que nous pouvons extraire.
            Plus tard, on pourra utiliser un deuxième algorithme ou améliorer l'expérience utilisateur en CLI ou avec une interface graphique.

         \subsection*{Créer un modèle de machine learning}
            Must be done:
            \begin{itemize}
               \item Extraire les données des \acrshort{smiles} [3]
               \item Utiliser l'algorithme SVM avec les \acrshort{smiles} [8]
               \item Comparer les résultats avec la version précédente [2]
               \item Documenter l'extraction des données et le nouveau modèle [1]
            \end{itemize}
            \subsection*{Tester d'autres algorithmes}
            Can be done:
            \begin{itemize}
               \item Essayer le deep learning [3]
               \item Essayer le clustering [3]
               \item ...
               \item Documenter les résultats [1]
            \end{itemize}
            \subsection*{Améliorer l'UX}
            Can be done:
            \begin{itemize}
               \item Créer une interface graphique [3]
               \item Créer une API [3]
               \item Améliorer le CLI [3]
               \item Créer une documentation pour les outils effectués [1]
            \end{itemize}

   \section{Défense du projet}
   \label{sec:defense}

         La défense est la présentation du projet qui intervient une semaine après la remise du rapport.

         \subsection*{Présentation du projet}
            Must be done:
            \begin{itemize}
               \item Créer une présentation [4]
            \end{itemize}