\chapter{Déclaration sur l'honneur}
\label{chap:honneur}
Je, soussigné, Simon Barras, déclare sur l'honneur que le travail rendu est le fruit d'un travail
personnel. Je certifie ne pas avoir eu recours au plagiat ou à toutes autres formes de fraudes.
Toutes les sources d'information utilisées et les citations d'auteur ont été clairement mentionnées.

\chapter{Remerciements}
\label{chap:remerciement}
Je tiens à Remercier M Wolf et son assistant M Donzallaz pour leur aide et leur disponibilité.
Ils m'ont aiguillé de nombreuses fois et m'ont permis de m'améliorer dans le domaine du \acrlong{ml}.
Je souhaite aussi remercier Mme Yerly qui a réalisé les bases du projets et m'a aidé de nombreuses fois à comprendre les données.
Pour finir, je remercie M Marti et son institut ChemTech pour la confiance accordé et pour la complétion et validation des données chimiques que je n'aurais en aucun cas pu faire moi-même.
J'apporte une mention spéciale à Github Copilot pour m'avoir aidé à écrire ce rapport et Simon Braillard pour la relecture du rapport.

\chapter{Logiciels utilisés}
\label{chap:logiciel}
Voici la version de pyhton ainsi que les librairies utilisées pour ce projet.

\begin{table}[ht]
    \centering
    \begin{tabular}{|l|l|}
    \hline
    \multicolumn{1}{|c|}{\textbf{Logiciel}} & \multicolumn{1}{c|}{\textbf{Version}} \\ \hline
    Python                                  & \multicolumn{1}{c|}{3.9.15}           \\ \hline
    aiohttp                                 & \multicolumn{1}{c|}{3.8.3}            \\ \hline
    matplotlib                              & \multicolumn{1}{c|}{3.6.3}            \\ \hline
    mordred                                 & 1.2.0                                 \\ \hline
    numpy                                   & 1.23.4                                \\ \hline
    openyxl                                 & 3.0.10                                \\ \hline
    pandas                                  & 1.5.1                                 \\ \hline
    pandas-profiling                        & 3.6.2                                 \\ \hline
    rdkit                                   & 2022.9.2                              \\ \hline
    scikit-learn                            & 1.1.3                                 \\ \hline
    \end{tabular}
    \captionof{table}{Logiciels utilisés pour ce projet}
\end{table}


